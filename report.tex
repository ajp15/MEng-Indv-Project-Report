\documentclass[12pt,twoside]{article}
\usepackage{setspace}
\usepackage[a4paper,hmargin=2.8cm,vmargin=2.0cm,includeheadfoot]{geometry}
\usepackage{indentfirst}

\usepackage{cite}
\usepackage{amsmath,amssymb,amsfonts}
\usepackage{algorithmic}

\usepackage[]{caption}
\usepackage{subcaption}

\usepackage{tabularx}
\usepackage{longtable}
\usepackage[]{multirow}
\usepackage{array} % to allow vertical alignment in tables
\usepackage{makecell} % allows multi-line tables
\usepackage{enumitem} % better enumeration

\usepackage{graphicx}
\usepackage{textcomp}
\usepackage{xcolor} % allows defining your own colours
\usepackage{chngcntr} % Make nice counter macros availalble
% 	\counterwithin{figure}{subsection} % Add Section Number to Figure Counter
% 	\counterwithin{table}{subsection} % Add Section Number to Table Counter
% \def\BibTeX{{\rm B\kern-.05em{\sc i\kern-.025em b}\kern-.08em
%     T\kern-.1667em\lower.7ex\hbox{E}\kern-.125emX}}

\usepackage{float} % enable forcing locations with [H], [T]

\usepackage{hyperref} % uncomment after document is finished to hyperlink everything
\hypersetup{
    colorlinks,
    linkcolor={red!50!black},
    citecolor={green!30!black},
    urlcolor={blue!80!black}
}


\begin{document}

\begin{titlepage}

\newcommand{\HRule}{\rule{\linewidth}{0.5mm}} % Defines a new command for the horizontal lines, change thickness here


%----------------------------------------------------------------------------------------
%	LOGO SECTION
%----------------------------------------------------------------------------------------

\includegraphics[width = 6cm]{./figures/imperial}\\[0.5cm] 

\begin{center} % Center remainder of the page

%----------------------------------------------------------------------------------------
%	HEADING SECTIONS
%----------------------------------------------------------------------------------------

\vspace{1.25cm}
\textsc{\Large Final Report for MEng Individual Project}\\[0.5cm] 
\textsc{\large Department of Bioengineering}\\[0.5cm] 
%----------------------------------------------------------------------------------------
%	TITLE SECTION
%----------------------------------------------------------------------------------------

\HRule \\[0.4cm]

\begin{spacing}{1.5}
{ \LARGE \bfseries Wireless Transmission of Electrochemical Signals for Early Diagnosis of Secondary Traumatic Brain Injury }\\
\end{spacing}

\HRule \\[1.5cm]

%----------------------------------------------------------------------------------------
%	AUTHOR SECTION
%----------------------------------------------------------------------------------------

%\begin{minipage}{0.4\hsize}
%\begin{flushleft} 
\large
\textit{Author:}
Aishwarya Pattar
%\end{flushleft}

%\begin{flushleft} 
\large
\textit{Supervisor:}
Prof. Martyn G. Boutelle
%\end{flushleft}


\vspace{5cm}
\small
\textit{Submitted in partial fulfilment of the requirements for the award of MEng in Biomedical Engineering from Imperial College London}

\end{center}

\vspace{2.5cm}
June 2019   \hfill  Word Count:

\vfill % Fill the rest of the page with whitespace



\makeatother


\end{titlepage}

\begin{abstract}
Traumatic Brain Injury (TBI) is caused by an impact to the head which causes damage to the brain. From the primary injury site, spreading depolarisation (SD) waves \cite{Leao1944} propagate to surrounding brain tissue and causes further damage. This can lead to secondary TBI, which can cause life-long disabilities or be fatal \cite{Hartings2011}. It is crucial to diagnose secondary TBI early and this can be achieved by continuously monitoring glucose, lactate, and potassium concentrations in the brain and analysing the quantitative data. If the glucose concentration transiently decreases and simultaneously both lactate and potassium concentrations transiently increase, this indicates the occurrence of an SD \cite{D.2010}.

This project aims to develop a prototype that can be integrated with a continuous online microdialysis system \cite{Rogers2017}, which records glucose, lactate, and potassium signals, and wirelessly transmits the information to an iPad application. A clinician can view the data in real time and identify if SDs are occurring in the patient's brain. The signals are displayed graphically on the app but the user is also able to obtain numerical data. The prototype was tested with standard signals as well as patient data which contained the characteristic transient changes indicating an SD. The app was successfully able to display voltage recordings in real time with an error of less than 4\%, and was capable of calculating the concentrations of glucose, lactate, and potassium in the brain from the electrochemical signals obtained. Future improvements are discussed to extend system performance and ultimately improve early diagnosis.

    
\end{abstract}

\newpage


\section{Introduction}
\subsection{Background}
Traumatic Brain Injury (TBI) occurs from an impact to the head resulting in an injury to the brain, commonly caused by road traffic accidents and falls \cite{Langlois2006}. The severity of the injury sustained to the brain can range from superficial swelling to oedema and haematomas, with higher severity of the TBI resulting in a higher mortality rate as shown in Table~\ref{table:severity of TBI}. Even if the injury does not result in death, TBI can lead to long term disability and cognitive problems \cite{WorldHealthOrganisation2006}. Approximately 2.5 million people visited the hospital for TBI related symptoms in the US during 2013, of which 56,000 resulted in death \cite{Taylor2017}. The prevalence of this injury incurs an estimated \$60 billion cost annually to society to cover medical costs and lost productivity \cite{Finkelstein2009}. 

Primary TBI is the damage sustained from the initial impact, whereas secondary TBI occurs from the cascade of biochemical processes following from primary TBI, resulting in damage to the tissue surrounding the primary injury site \cite{Norton2008}. Secondary TBI tends to occur after a delayed period of time in about 30\%--40\% of cases, however whether it occurs is unpredictable and unpreventable \cite{Pagkalos2017}. Secondary effects, such as oedema, hypoxia and ischemia, can lead to neural degeneration and irreversible damage which can cause severe disability or be fatal \cite{Murthy2005}. Hence, it is crucial to diagnose secondary TBI as early as possible and provide appropriate medication to treat and minimise the damage.

\begin{table}[H]
\centering
\begin{tabular}{||c c||} 
 \hline
 Severity of TBI & Mortality (\%) \\ [0.5ex] 
 \hline\hline
 Mild & \textless 1 \\ 
 Moderate & 2-5 \\
 Severe & 20-50 \\
 \hline
\end{tabular}
\caption{Severity of TBI affecting mortality rates \cite{WorldHealthOrganisation2006}.}
\label{table:severity of TBI}
\end{table}


When a patient is admitted with a head injury, common methods for prognosis include the Glasgow Coma Scale (GCS) \cite{WorldHealthOrganisation2006} and the Glasgow Outcome Scale (GOS) \cite{Jennett1975}, which use a scoring system based on the patient's physiological state. The GCS assesses ocular, motor, and verbal responses to assign the patient with a score out of 15 to determine the severity of TBI. The GOS categorises the patient into one of five categories that predicts the recovery of the patient. These methods are commonly accompanied with a neuroimaging of the brain to provide further information. The main disadvantage of these methods is that they are subjective, and it is difficult to identify mild symptoms of TBI which are displayed subtly as non-specific symptoms \cite{Bettermann2012}. Furthermore, these methods only capture information at a specific period in time meaning that if symptoms occur in between check ups, they will be unobserved until the next appointment. 

Prognostic models \cite{Steyerberg2008} have been developed from the IMPACT (International Mission for Prognosis and Analysis of Clinical Trials in TBI) database \cite{Maas2007} that take into account several patient characteristics, such as age, hypotension, and haemoglobin levels, that predict the patient's outcome after six months using odds ratios. These high complexity models allow for clinicians to plan treatment based on the prognosis, however it is still not a method to diagnose a patient, merely predict the outcome. 

Early diagnosis is crucial for the best recovery, and to do so, there is a need to continuously monitor the patient and receive quantitative data in real time for accurate diagnosis. The Boutelle Research Group have been investigating utilising brain signals that are associated with TBI. From the primary injury site, spreading depolarisation (SD) waves \cite{Leao1944} propagate into surrounding brain tissue and cause secondary damage \cite{Brain2011}. Research has shown that monitoring SDs provides more information than routine clinical checkups \cite{Hartings2011}. An SD is characterised by the cells in the injury sites having a high energy demand to repolarise the membrane potential, hence the glucose concentration in the extracellular fluid transiently decreases by approximately 18-93$\mu M$. The cells undergo anaerobic respiration during this period, so the extracellular lactate concentration transiently increase by 5-100$\mu M$ \cite{D.2010}. Glucose and lactate have been declared by the International Microdialysis Collaborative Group as the most clinically useful signals for the prognosis of TBI \cite{Hutchinson2015} so these transient changes are useful to monitor in high time resolution. Furthermore, during the same transient period, the potassium ion concentration increases \cite{Rogers2011} from 3$mM$ \cite{Katzman1976} to 30-50$mM$ \cite{Ayata2015} due to the cells depolarising. 


\begin{figure}[t!]
\centering
\includegraphics[trim={0cm 5cm 0.5cm  5cm}, clip, width=1\textwidth]{./figures/conc.pdf}
\captionsetup{justification=centering}
\caption{Electrochemical changes during a spreading depolarisation occurrence at 10 minutes. K+ and lactate show an increase in concentration whilst glucose shows a decrease in concentration \cite{Rogers2017}. Measurements obtained from continuous online microdialysis.}
\label{fig: SD}
\end{figure}

\subsection{Online Microdialysis System}
The research group have developed a novel monitoring system, a continuous online microdialysis (coMD), to monitor the electrochemical changes that occur during a SD wave with excellent temporal resolution. A miniaturised microdialysis probe is inserted on the surface of the brain and is perfused with sterile artificial cerebrospinal fluid (aCSF). The microdialysite then perfuses through a microfluidic chip that houses the three sensors that detect the glucose, potassium and lactate concentrations in the brain. This set up allows for online microdialysis monitoring and the transient changes in glucose, lactate and potassium to be accurately measured \cite{Rogers2017} as shown in Figure~\ref{fig: SD}.

To collect data from the patient, the microfluidic analysis system is placed behind the patient's bed on a trolley. The sensors are then wired into the Powerlab data acquistition hardware, which is then wired to a computer so that the measurements from the sensors could be viewed in real time on the associated software, PowerLab Pro \cite{Rogers2017}. This setup is very bulky, restricts patient movement, is liable to the wires breaking, and the wires may introduce noise \cite{Ferguson2011}. A more desirable system would be miniaturised and allow for wireless data transmission.



\subsection{Aims}
This project aims to develop a wireless communication system between the coMD system and an iOS application. Information from the three electrochemical sensors will be received simultaneously in an iPad application in real time to allow for continuous monitoring of the patient in high temporal resolution. The app must be user friendly and be of high quality as it aims to be used by a clinician to monitor the state of the patient. This can be achieved by displaying the data graphically to allow for the transient changes to be identified. It is desirable for the app to contain features that allow for easy access to and readability of the data. Therefore the clinician should be able to access numerical data as well as graphical. It is important that the signals are transmitted and displayed accurately so suitable processing methods need to be implemented. 


\newpage

\section{Prototype Design}
The prototype developed intends to be integrated with a PCB currently being developed by the research group (see \textbf{Appendix 1}). The PCB consists of five potentiometric channels, allowing for the measurement of potassium ion concentration, and four amperometric channels to allow for glucose and lactate concentration measurements. Whilst the amperometric channels record current, this is then converted into a digital voltage form. An ADS1298, which is a 24-bit resolution analogue-to-digital converter (ADC), receives the input measurements simultaneously and discretises and amplifies the signals \cite{TexasInstruments2010}. An ATmega328p microcontroller is used and the PCB allows for both SPI (serial peripheral interface) and UART (universal asynchronous receiver-transmitter) data transmission capabilities.

The prototype developed is a simplified model of the PCB. The prototype receives three amplified voltage signals, so it assumes the current values from the lactate and glucose sensors have been converted into voltage, and all the signals have been amplified. The prototype then extends further from the functionality of the PCB by developing the wireless transmission of the data and the development of an iPad application, as shown in the flowchart in Figure~\ref{fig: flowchart}. The code for this project can be found in  \textbf{Appendix 2}.

\begin{figure}[H]
\centering
\includegraphics[trim={0cm 6cm 0.5cm  2cm}, clip, width=1\textwidth]{./figures/Flowchart.pdf}
\captionsetup{justification=centering}
\caption{Flowchart of the key elements in the prototype system}
\label{fig: flowchart}
\end{figure}



\subsection{Hardware}
The prototype of the hardware consists of an Arduino Nano and a Adafruit Bluefruit LE SPI Friend, as shown in Figure X. The Arduino Nano was chosen as a suitable microcontroller for prototyping and mimicking the PCB as it also contains an ATmega328p microcontroller and has an inbuilt ADC. However, the ADC is only 10-bit, so it has poorer precision than the ADS1298. The Arduino is also simple to programme due to the IDE and libraries available, making it suitable for prototyping.

The Arduino Nano receives three input voltage signals ranging from 0V to 5V at the analogue pins A0, A1, and A7. The signals are discretised by the inbuilt ADC. The nano records the time at which each signal is read, along with the value of the signal. The signal recordings are processed before being passed to the Bluefruit.

% Insert Figure of Breadboard (made on Fritzing)%




\subsection{Signal Processing}
Every 200ms the Arduino reads the input signals at the A0, A1 and A7 pins and records the time at which sampling occurred. The recorded data is mapped to its corresponding voltage value. For every five samples of data recorded, so in a 1s period, the average voltage is calculated for each signal. This averaging methods allows large deviations and spikes to be smoothed, thereby reducing the effect of noise on the outputted data.

Each input signal received by the Arduino is filtered with a low-pass filter with a cut off frequency of 1Hz. This cut off frequency was chosen because the events indicating a SD occur over a long time frame, as shown in Figure~\ref{fig: SD}, hence the changes occur slowly and at low frequency. The low pass filter attenuates high frequency components that may arise from noise or patient movement, allowing for a cleaner signal to be obtained that contains the relevant data.







\subsection{Wireless Transmission}
Bluetooth Low Energy (BLE) and ZigBee are both popular wireless data transmission options for IoT projects. Table~\ref{table:BLE vs ZigBee} summarises the differences between BLE and ZigBee.

\begin{table}[H]
\centering
\begin{tabular}{||c c||} 
 \hline
 BLE & ZigBee \\ [0.5ex] 
 \hline\hline
 Short range (77m) & Medium range (291m) \\
 Higher data rate: 1Mbps bursts & Lower data rate: 250kbps \\ 
 PAN (personal area network) & LAN (local area network) \\
 Throughput: 0.27Mbps & Throughput: 0.03Mbps \\
 Latency: 3-6ms & Latency: 15ms \\
 Sleeps between bursts $\rightarrow$ uses less power & No sleep functionality \\
 Supported on most OSs including iOS & Not supported on most OSs \\
 \hline
\end{tabular}
\caption{BLE vs. ZigBee \cite{Ray2015, Christiano}}
\label{table:BLE vs ZigBee}
\end{table}

For this project, BLE is the most suitable option. BLE offers more efficient data transfer and is compatible with iOS. BLE has a shorter range than ZigBee, but this is not of hindrance as the patient will remain close to the iPad that will receive the signals. BLE is commonly used for health and fitness trackers, making it a suitable method of transmission for this project's application.

Adafruit provide a variety of Bluetooth low energy modules that are compatible with Arduino and provide a library that can be used with the Arduino IDE: Adafruit\_BluefruitLE\_nRF51. Adafruit also provide an iOS app to test the bluetooth module and view the received signals. Whilst the PCB currently being developed allows for both SPI and UART data transmission, SPI was the chosen mode for the prototype hence the Bluefruit LE SPI Friend was chosen over the Bluefruit LE UART Friend. Both share similarities and have the same library that is used in the Arduino IDE, however SPI was preferred as it has a clock line where data can be sent along with the clock pulses, so the timing of when data is sent is more reliable. The Arduino Nano has inbuilt SPI pins: D11 is MOSI, D12 is MISO and D13 is SCK. These connect to the corresponding pins on the Bluefruit. D10 on the Arduino is programmed as the CS and D7 is programmed as the IRQ, which are then wired to the corresponding Bluefruit pins. 

The Adafruit\_BluefruitLE\_nRF51 library provides a header file specifically for the Bluefruit LE SPI Friend which provides functions that allow for the transmission of data via BLE to a paired device, such as {\tt{ble.print()}}. The limitation of this function is that a maximum of 20 characters can be sent at a time otherwise the data becomes fragmented. This means that the data from each input pin and the corresponding time of recording has to be wirelessly transmitted one at a time rather than all three simultaneously to prevent the risk of the data fragmenting. The data from each input pin is outputted in characters as:

\begin{align}
    Output = ID + Time + Value + \backslash n.
    \nonumber
\end{align}

The ID at the beginning identifies which signal is being read i.e. G for glucose, and the newline character ($\backslash$n) at the end concludes the data transmission. If both these identifiers are not present at the receiving end, this indicates that the data is fragmented and the receiving app will discard the data. 

Another limitation was observed when all three inputs were periodically sent within a 200ms period. The data would initially be sent correctly but after some time, the data would fragment. The Bluefruit struggled to send data so quickly, which created a backlog of data that was responsible for the fragmentation. Eventually, the Bluetooth module would disconnect from the app. To prevent this, the three signals were periodically sent every 1s, spaced such that potassium was sent at 200ms, glucose at 600ms, and lactate on the 1s. Since the electrochemical changes in the brain occur at slow rates, for example an SD event takes about 10 minutes, a sampling speed of 1s is suitable for this application. 





\subsection{iPad Application}
The iPad app was written in Swift using XCode. The story board is shown in Figure X.

\newpage

\section{Testing}
\subsection{Standard Signals}

\subsection{Brain Signals}
\newpage

\section{Results}
\subsection{Standard Signals}

\begin{figure}[h!]
\centering
\includegraphics[trim={0cm 0cm 0cm  0cm}, clip, width=.75\textwidth]{./figures/standardsignals/picologChemComp.pdf}
\captionsetup{justification=centering}
\caption{Standard signals recorded using a PicoScope. Green: noise, red: sine wave, blue: square wave.}
\label{fig: test1 picolog}
\bigbreak
\includegraphics[trim={0cm 0cm 0cm  0cm}, clip, width=.75\textwidth]{./figures/standardsignals/appChemComp.pdf}
\captionsetup{justification=centering}
\caption{Standard signals recorded on iPad app. Green: noise, red: sine wave, blue: square wave.}
\label{fig: test1 app}
\end{figure}



Figure~\ref{fig: test1 picolog} shows the raw signals received on PicoLog, and Figure~\ref{fig: test1 app} shows the same three signals received on the Chemical Composition page of the app after undergoing filtering and processing on the Arduino.


\subsection{Spreading Depolarisation}

\begin{figure}[h!]
\centering
\includegraphics[trim={0cm 0cm 0cm  0cm}, clip, width=0.975\textwidth]{./figures/test2.pdf}
\captionsetup{justification=centering}
\caption{Left: Raw patient data. Right: Signals received on app. Top to bottom: Glucose, Lactate, Potassium. Plots of voltage against time.}
\label{fig: test2}
\end{figure}

Figure~\ref{fig: test2} shows the raw patient signals, before downsampling and processing, compared to the signals received on the app. Figure~\ref{fig: test2 conc} shows the concentration vs. time graphs produced on the app calculated from the inputted calibration values.

\begin{figure}[p]
\centering
\includegraphics[trim={0cm 0cm 0cm  0cm}, clip, width=.6\textwidth]{./figures/patientsignals/Gconc.pdf}
\bigbreak
\includegraphics[trim={0cm 0cm 0cm  0cm}, clip, width=.6\textwidth]{./figures/patientsignals/Lconc.pdf}
\bigbreak
\includegraphics[trim={0cm 0cm 0cm  0cm}, clip, width=.6\textwidth]{./figures/patientsignals/Kconc.pdf}
\captionsetup{justification=centering}
\caption{Concentration (mM) vs. time graphs calculated from calibration values. Basal concentration and transient concentration change during an SD are shown.}
\label{fig: test2 conc}
\end{figure}
\newpage

\section{Discussion}
\subsection{Standard Signals}

The raw square wave, shown in Figure~\ref{fig: test1 picolog}, has an instantaneous change between the two voltage levels. However, on the app, the discontinuities were not as sharp because when a discontinuity occurs, the averaging method implemented on the Arduino softens this sudden change. The Arduino reads in five input signal per second and the mean value is calculated and sent to the app every second. Whilst the square wave cannot be replicated exactly, this issue doesn't affect the neurochemical signals. As seen in Figure~\ref{fig: SD}, any sharp discontinuities that occur are due to noise or electrical artefacts, so the averaging method implemented will smooth these out. The transient changes that occur during an SD for potassium, glucose and lactate occur gradually so the shape will be captured. Whilst the square wave was meant to have a peak-to-peak voltage of 1V, it can be observed that the upper bound exceeds 1V. However, this is a fault of the signal generator as this is also observed on the PicoScope. The prototype successfully followed the signal it was given.

The effect of averaging and filtering noise and artefacts can be shown by comparing the noise signals in Figures~\ref{fig: test1 picolog} and \ref{fig: test1 app}. The raw noise signal displayed on PicoLog shows lots of sudden peaks. On the other hand, the app shows the noise signal has been smoothed. 

Comparing the sine waves, the app shows a slightly more jagged signal. This is because the app receives the signals at a lower frequency of 1Hz, compared to the PicoScope's 5Hz sampling frequency. The higher frequency captures the signal better so results in a smoother wave. Regardless, the sine wave is well represented on the app and this won't be an issue for electrochemical signals as changes occur over a longer duration. 

The peak amplitudes of the sine and square wave on the app match that recorded on PicoLog. However, it can be noted that the app has a time lag of 1.25s compared to the PicoScope because the wireless transmission is not actually negligible as assumed in section \ref{section: real time}. However, this time lag is a systemic bias and is kept constant across all time values, and is much smaller compared to the timescale of an SD so is not significant to the end user.



\subsection{Spreading Depolarisation}

Figure~\ref{fig: test2} shows how the app was able to respond to receiving real data as opposed to standard signals. For all neurochemicals, the app was able to capture the shape of the signals even with a prototype's lower sampling rate of 5Hz, compared to the raw data's sampling frequency of 200Hz. The averaging method and filtering implemented on the Arduino was able to reduce the noise levels of the signals, seen clearly in the glucose signal where the variation in the signal is reduced, and also remove sharp noise in the case of the lactate signal. Sharp noise was still present in the glucose and potassium signals, however the app has been able to attenuate the noise to a lower amplitude.

Figure~\ref{fig: test2 conc} shows the conversion of voltage to concentration based on the calibration voltages, given in Table~\ref{table: calibration conc}, that were inputted on the Chemical Composition page of the app. The experiment shows that glucose decreased by $\sim0.23 mM$, lactate increased by $\sim2.7 mM$, and potassium increased by $\sim0.78 mM$. These results appear to deviate greatly from literature which expected glucose to decrease by 18-93$\mu M$ and lactate to increase by 5-100$\mu M$. However, the patient that this data set was obtained from displayed changes of $-0.223mM$, $2.7mM$, and $0.78mM$ (corresponding to glucose, lactate, and potassium) \cite{Rogers2017}, hence the app was able to calculate the concentration changes well given the calibration values provided. Furthermore, the patient's baseline concentrations were recorded as 0.544$mM$, 0.068$mM$, and 2.71$mM$ respectively for glucose, lactate, and potassium \cite{Rogers2017}. The baseline concentrations have been captured by the app and can be observed just before the SD transient change occurs.



\subsection{iPad Application}
The app is successfully able to support receiving and plotting three independent signals simultaneously, and display patient data clearly. The app has a simple and professional design with clear navigation. The graphs take up a large portion of the screen, allowing the data to be viewed clearly. To see finer detail, the graph can be zoomed and exact values of individual data points can be viewed by clicking on a point. The app shows all plots superimposed on one graph to allow for the transient decrease in glucose and transient increase of lactate and potassium that characterise a spreading depolarisation to be observed. However, if the user wishes to see an individual plot in detail, this too is allowed on the individual concentration pages shown on pages 3a, 3b and 3c in Figure~\ref{fig: storyboard}. These individual plots allow for the data to be viewed in terms of voltage or concentration, and the app prompts the user with instructions to view the concentration graph. These features allow the app to be user friendly and clearly present all the relevant data for a clinician to view.




\subsection{Limitations and Future Improvements}
A major limitation of the prototype is the components used in the hardware. The inbuilt ADC on the Arduino only has 10-bit resolution and can only detect voltage readings between 0V and 5V. This gives a resolution of 4.88mV. An ADS1298 has 24-bit resolution, which allows to capture the neurochemical signals more precisely. For the same range of voltages, the resolution would be 0.298$\mu$V. The Arduino input pins also cannot receive input voltage outside of 0V to 5V. Whilst the neurochemical signals are not expected to exceed 5V, the potassium input can produce negative voltages, for example during calibration. These negative voltages need to be captured for the signals to be of clinical relevance. These issues will be resolved when the PCB being developed by the research group is used instead of the Arduino.

Furthermore, the Bluefruit LE SPI Friend's limitations include the transmitted string being restricted to 20 character, and the speed of transmission used in the prototype was restricted to 1 sample per second for each neurochemical signal. Ideally, all signal that the Arduino records should be sent to the app so that the app receives 5 samples per second. This would mean that no raw readings will be lost and an averaging method could be implemented on the app instead of on the Arduino. To achieve this, a BLE module that can send more characters at once, or one that is capable of sending data with smaller time intervals between transmission is required. Alternatively, the transmitted data could be encoded in a numerical binary format rather than as a string to reduce the size of transmission payload. 

A limitation of implementing a block average is that the time is not exact as it represents a set of data instead of an individual data point. When the average value is transmitted, the time wirelessly sent with the data is the time of the latest reading. If the app receives all the raw signals and a moving average filter is implemented on the app, then this would resolve the issue as the average is calculated around the time point.

Currently, the app only receives data whilst it is open and connected to the BLE module. To improve the usability of the app, the Arduino or ATMega328p on the PCB could store data in persistent storage until a connection is made and then transfer the data to the app. This is useful in case the Bluetooth module loses connection with the app. However, with the current BLE module, sending a large store of data when the connection is made to the app may create a backlog of data and risk causing the Bluefruit to disconnect. A different BLE module would need to be used and the transmission process optimised for this purpose.

To further smooth out noise in the signals, further processing could be implemented on the app once the data is received. This could involve looking for sudden peaks that then return back to a similar value as before the peak occurred. 

To make the app more user friendly and improve its functionality, the app could be encoded to detect when calibration is occurring. This would eliminate the need for the clinician the manually insert the calibration values into the app. Furthermore, the app could be programmed to identify when an SD occurs, by identifying the transient changes that characterise an SD, and send a push notification or alert to inform the clinician. Highlighting potential SD waves would allow for the clinician to quickly identify points of concern and allow for quicker diagnosis rather than the clinician manually analysing the data. However, the expected change in concentration during an SD can vary by a large amount i.e. glucose transiently decreases by approximately 18-93$\mu M$ and lactate transiently increases by 5-100$\mu M$ \cite{Rogers2017}. Using a large data set of different SDs and machine learning, a model could be trained to effectively identify an SD occurrence in real time. This model could then be integrated into the app.


\subsection{Conclusion}
The prototype has been designed such that the firmware features can be integrated with the PCB being developed, which can then be used with the iPad app. The app was able to receive and represent standard signals and spreading depolarisations, and accurately calculate the corresponding concentration changes.
\newpage

\section{Conclusion}
The prototype has been designed such that the firmware features can be integrated with the finished PCB developed by the Boutelle group. This allows for wireless transmission between the PCB and the app. The prototype was successfully able to transmit three signals simultaneously and display them in real time on the app. The app was able to clearly display the received signals and SD waves graphically. Users are able to access numerical data from the graphs which allows clinicians to conduct more detailed analysis. Finally, the prototype was successfully able to provide information of clinical relevance by allowing accurate conversion of voltage signals to concentration.

\newpage

\section{Acknowledgements}
I would like to thank Professor Martyn Boutelle for allowing me to pursue this project that has greatly diversified my skill set, and for guiding me through this project. I would also like to thank Dr. Michelle Rogers for helping me navigate through LabChart and understand the data. I would like to thank Aidan Wickham for providing me with details about his PCB, which this project is based on. I would also like to thank Georgios Zafeiropoulos and Konstantinos Petkos for guiding me through the best methods to validate and test my prototype. 

Finally, I would like to thank my parents, Dr. Jayaprakash Pattar and Dr. Jyothsna Murthy, for believing in me more than I could ever believe in myself, and Rohan Padmanabhan for always supporting me through the highs and lows of this project.

\newpage
\bibliography{library.bib}
\bibliographystyle{unsrt}

\newpage

\appendix

\section{Appendix: Project Code}
\label{appendix: a}
\begin{itemize}
    \item Firmware code can be found at: \newline \url{https://github.com/ajp15/MEngIndvProject-arduino}
   \item iPad app development can be found at: \newline \url{https://github.com/ajp15/MEngIndvProject-app2}
\end{itemize}


\end{document}
