The prototype developed intends to be integrated with a PCB currently being developed by the research group. The PCB consists of five potentiometric channels, allowing for the measurement of potassium ion concentration, and four amperometric channels to allow for glucose and lactate concentration measurements. The potentiometric readings are amplified using an instrumental amplifier, and the amperometric readings are amplified using a transimpedance amplifier, which also converts the current to voltage. An ADS1298, which is a 24-bit resolution analogue-to-digital converter (ADC), receives the input measurements simultaneously and discretises and further amplifies the signals \cite{TexasInstruments2010}. An ATmega328p microcontroller is used and the PCB allows for both SPI (serial peripheral interface) and UART (universal asynchronous receiver-transmitter) data transmission capabilities. The PCB design in summarised in the block diagram in Figure~\ref{fig: PCB block diagram}.

\begin{figure}[h!]
\centering
\includegraphics[trim={0cm 20.5cm 0.5cm  2.9cm}, clip, width=1\textwidth]{./figures/CircuitBlockDiagram.pdf}
\captionsetup{justification=centering}
\caption{Block Diagram of the PCB being developed by the Boutelle Research Group for continuous monitoring of TBI markers}
\label{fig: PCB block diagram}
\end{figure}

\begin{figure}[t!]
\centering
\includegraphics[trim={0cm 19.5cm 0.5cm  2.5cm}, clip, width=1\textwidth]{./figures/Flowchart.pdf}
\captionsetup{justification=centering}
\caption{Flowchart of the key elements in the prototype system}
\label{fig: flowchart}
\end{figure}

The prototype developed is a simplified model of the PCB. The prototype receives three amplified voltage signals, corresponding to glucose, lactate and potassium, so it continues on from the PCB block diagram after the ADS1298. The prototype then extends further from the functionality of the PCB by processing the received signals, developing the wireless transmission of the data and developing an iPad application, as shown in the flowchart in Figure~\ref{fig: flowchart}. The code for this project can be found in Appendix~\ref{appendix: a}.



\subsection{Hardware}
The hardware of the prototype consists of an Arduino Nano and a Adafruit Bluefruit LE SPI Friend, as shown in Figure~\ref{fig: breadboard}. The Arduino Nano was chosen as a suitable microcontroller for prototyping and mimicking the PCB as it also contains an ATmega328p microcontroller and has an inbuilt ADC. However, the ADC is only 10-bit, so it has poorer precision than the ADS1298. The Arduino is also simple to programme due to the IDE and libraries available, making it suitable for prototyping.

\begin{figure}[H]
\centering
\includegraphics[trim={0cm 10.5cm 0cm  10cm}, clip, width=1\textwidth]{./figures/Breadboard2.pdf}
\captionsetup{justification=centering}
\caption{Prototype assembled on breadboard}
\label{fig: breadboard}
\end{figure}

The Arduino Nano receives three input voltage signals ranging from 0V to 5V at the analogue pins A0, A1, and A7. These input pins correspond to glucose, lactate, and potassium. The signals are discretised by the inbuilt ADC. The Nano records the time at which each signal is read, along with the value of the signal. The signal recordings are processed before being passed to the Bluefruit.



\subsection{Firmware}
The Arduino Nano is encoded to sample and process the three input signals received, then to control the Bluefruit to wirelessly transmit the signals via Bluetooth low energy (BLE). The algorithm encoded in the Arduino is described in Figure~\ref{fig: psuedocode}.


\subsubsection{Signal Processing}
Every 200ms the Arduino reads the input signals at the A0, A1 and A7 pins and records the time, in milliseconds since the Arduino was connected to power, at which sampling occurred. Each input signal received by the Arduino is filtered with a low-pass filter with a cut off frequency of 1Hz. This cut off frequency was chosen because the events indicating a SD occur over a long time frame, as shown in Figure~\ref{fig: SD}, hence the changes occur slowly and at low frequency. The low pass filter attenuates high frequency components that may arise from noise or patient movement, allowing for a cleaner signal to be obtained.

For every five samples of data recorded, so in a 1s period, the average voltage is calculated for each signal. This averaging methods allows large deviations and spikes to be smoothed, thereby reducing the effect of noise on the outputted data.

\begin{figure}[t!]
\centering
\includegraphics[trim={1cm 5cm 1cm  4cm}, clip, width=1\textwidth]{./figures/psuedocode.pdf}
\captionsetup{justification=centering}
\caption{Flowchart of algorithm encoded in Arduino Nano}
\label{fig: psuedocode}
\end{figure}



\subsubsection{Wireless Transmission}
BLE and ZigBee are both popular wireless data transmission options for IoT projects. Table~\ref{table:BLE vs ZigBee} summarises the differences between BLE and ZigBee. For this project, BLE is the most suitable option. BLE offers more efficient data transfer, shown by the lower latency time and higher throughout, and is compatible with iOS. A disadvantage of BLE is that it has a shorter range than ZigBee, however this is not of hindrance as the range is sufficient for the application since the patient will remain close to the iPad that will receive the transmitted signals. BLE is commonly used for health and fitness trackers, making it a suitable method of transmission for this project's application.

\begin{table}[h!]
\centering
\begin{tabular}{||c c||} 
 \hline
 BLE & ZigBee \\ [0.5ex] 
 \hline\hline
 Short range (77m) & Medium range (291m) \\
 Higher data rate: 1Mbps bursts & Lower data rate: 250kbps \\ 
 PAN (personal area network) & LAN (local area network) \\
 Throughput: 0.27Mbps & Throughput: 0.03Mbps \\
 Latency: 3-6ms & Latency: 15ms \\
 Sleeps between bursts $\rightarrow$ uses less power & No sleep functionality \\
 Supported on most OSs including iOS & Not supported on most OSs \\
 \hline
\end{tabular}
\caption{BLE vs. ZigBee \cite{Ray2015, Christiano}}
\label{table:BLE vs ZigBee}
\end{table}

Adafruit provide a variety of Bluetooth low energy modules that are compatible with Arduino, and they provide a library that can be used with the Arduino IDE: Adafruit\_BluefruitLE\_nRF51. Adafruit also provide an iOS app to test the Bluetooth module and view the received signals. Whilst the PCB currently being developed allows for both SPI and UART data transmission, SPI was the chosen mode for the prototype hence the Bluefruit LE SPI Friend was chosen over the Bluefruit LE UART Friend. Both share similarities and use the same library in the Arduino IDE, however SPI was preferred as it has a clock line where data can be sent along with the clock pulses, so the timing of when data is sent is more reliable. Furthermore, the Bluefruit LE SPI Friend is smaller in size than the Bluefruit LE UART Friend, so allows the prototype to be miniaturised further. 

The Arduino Nano has inbuilt SPI pins: D11 is MOSI, D12 is MISO and D13 is SCK. These connect to the corresponding pins on the Bluefruit, which are shown by the orange connections in Figure~\ref{fig: breadboard}. D10 on the Arduino is programmed as the CS and D7 is programmed as the IRQ, which are then wired to the corresponding Bluefruit pins, shown by the blue wires in Figure~\ref{fig: breadboard}. 

The Adafruit\_BluefruitLE\_nRF51 library provides a header file specifically for the Bluefruit LE SPI Friend which provides functions that allow for the transmission of data via BLE to a paired device. The function used is {\tt{ble.print()}}. The limitation of this function is that a maximum of 20 characters can be sent at a time otherwise the data becomes fragmented. This means that the data from each input pin and the corresponding time of recording has to be wirelessly transmitted one at a time rather than all three sent simultaneously to prevent the risk of the data fragmenting. The data from each input pin is outputted in characters as:

\begin{align}
    Output = ID + Time + Voltage + \backslash n.
    \nonumber
\end{align}

The \textit{ID} at the beginning identifies which signal is being sent i.e. G for glucose, and the newline character (\textit{$\backslash$n}) at the end concludes the the string being transmitted. If both these identifiers are not present at these specific locations when received on the app, this indicates that the data is fragmented and the app will discard the data. 

Another limitation was observed when all three inputs were periodically sent within a 200ms period. The data would initially be sent correctly but after some time, the data would fragment. The Bluefruit struggled to send the data so quickly, which created a backlog of data that was responsible for the fragmentation. Eventually, the Bluetooth module would disconnect from the app. To prevent this, the three signals were periodically sent every 1s, spaced such that potassium was sent at 200ms, glucose at 600ms, and lactate on the 1s as highlighted in Figure~\ref{fig: psuedocode}. Since the electrochemical changes in the brain occur at slow rates, for example an SD event takes about 10 minutes, a sampling speed of 1s on the app is suitable for this application. 





\subsection{iPad Application}
The iPad app was written in Swift using XCode. The app layout is shown in Figure~\ref{fig: storyboard}. The three pages shown in section 3 of Figure~\ref{fig: storyboard} all have the same layout. Details of this layout can be seen clearly on the Potassium Ion Concentration page, and the pages Glucose Concentration and Lactate Concentration have the same features. 

Once the user logs in, the app connects to the Bluefruit LE SPI Friend and receives the strings of data sent by the Bluefruit and checks that the \textit{ID} and newline character are in place. Depending on the \textit{ID}, the app stores the \textit{Time} and \textit{Voltage} string in arrays corresponding to the electrochemical signal being read, i.e. if the \textit{ID = G}, the \textit{Time} is appended to an array called  {\tt{Gtime}} and \textit{Voltage} is appended to an array called {\tt{Gvoltage}}. Each new data string for a given electrochemical signal is received every 1s. The \textit{Time} string received is converted from the time in milliseconds into the real clock time at which the signal was read from the Arduino analogue input pin. It is important that the signals are timestamped with the real time of recording as opposed to the time at which the signals were received on the app because over time there may be a time lag in transmission.

\begin{figure}[p]
\centering
\includegraphics[trim={1cm 2.5cm 1cm  2cm}, clip, width=1\textwidth]{./figures/appStoryboard3.pdf}
\captionsetup{justification=centering}
\caption{Storyboard of the iPad application. 1) Log in page. 2) Chemical Composition page. 3a) Potassium Ion Concentration page. 3b) Glucose Concentration page. 3c) Lactate Concentration page. Pages 3a, 3b, and 3c have the same layout and are accessed from Page 2.}
\label{fig: storyboard}
\end{figure}


\subsubsection{Charts API}
To convey the information to the clinician clearly, the data received was displayed graphically using the Charts API developed by Daniel Gindi. The API allowed for basic graphs to be fully customised based on the needs of this project. The graphs display the voltage recorded from an input pin against the real clock time of the reading for each electrochemical signal. The app allows for all three signals to be viewed on one graph, as seen on the Chemical Composition page in Figure~\ref{fig: storyboard}, or as individual plots shown in pages 3a, 3b and 3c. These individual plots can be accessed by clicking on the graph legends on the Chemical Composition page, located above the graph. All graphs can be scaled and are zoombale, allowing to clearly view specific data. Furthermore, clicking on a specific point on a plot displays a grey bubble at the location which shows the corresponding x and y value at that point. This feature can be seen on the Chemical Composition page but works across all plots on all pages.


\subsubsection{Calibration}
In the individual concentration plots on pages 3a, 3b, and 3c in Figure~\ref{fig: storyboard}, in the top right corner of the graph there is a segmented control button that has \textit{Voltage} on the left and \textit{Concentration} on the right. Selecting a button allows to view the graph with either voltage or concentration on the y-axis. To convert the voltage values received into the corresponding concentration of the neurochemical signal in the brain, the relationship between the two needs to be known. This is obtained through calibration. 

\begin{table}[h!]
\centering
\begin{tabular}{||c c c||} 
 \hline
 [K+] (mM) & [Glucose] (mM) & [Lactate] (mM) \\ [0.5ex] 
 \hline\hline
 2.70 & 0.00 & 0.00 \\
 6.35 & 1.00 & 0.50 \\
 10.00 & 2.00 & 1.00 \\
 \hline
\end{tabular}
\caption{Known concentrations of solutions used for calibration}
\label{table: calibration conc}
\end{table}

During calibration, known concentrations of potassium ions, glucose and lactate, shown in Table~\ref{table: calibration conc}, are provided to the microdialysis system and the outputted voltage is recorded. Whilst monitoring a patient, calibration occurs at frequent intervals as the sensors degrade over time, hence voltage output for the same known concentration will change with time. The Chemical Composition page of the app allows for the calibration values to be typed into the text field. Once the calibration voltages have been inputted, the input voltages can be converted into concentration when the \textit{Concentration} button is clicked on the segmented control, and the graph displays concentration against real time. If calibration values aren't provided and the \textit{Concentration} button is clicked, the app will prompt the user to insert the calibration voltages on the Chemical Composition page. \newline 

\noindent \underline{Potassium Calibration} 

For potassium, the conversion from voltage to concentration is given by a logarithmic relationship and can be derived by considering an ion selective electrode. \textbf{** NEED TO COMPLETE **} \newline 

\noindent \underline{Glucose and Lactate Calibration} 

The conversion from voltage to concentration for glucose and lactate is derived from the enzymatic catalysed reaction that occurs on the electrode surface of the sensors \cite{Patel:2011:10.1016/j.bios.2010.11.033}, and can be obtained by starting from the Hill equation. Since the sensors measure current, the Hill equation is written as

\begin{equation}
    I_{rate} = \frac{I_{max}}{1 + \big( \frac{K_{m}}{[S]} \big)^{n}},
    \label{equation: Hill equation}
\end{equation}

\noindent where $I_{rate}$ is the reaction rate, $I_{max}$ is the maximum reaction rate, $[S]$ is the substrate concentration, i.e. glucose or lactate concentration, $K_{m}$ is the substrate concentration that results in a rate that is $\frac{I_{max}}{2}$, and $n$ is the Hill coefficient which is a measure of cooperativity between the substrate and binding site.

The Hill coefficient is equal to 1 since glucose and lactate have no substrate binding cooperativity. Therefore the Hill Equation can be reduced to the Michaelis-Menten equation, which is given by:

\begin{equation}
    I_{rate} = \frac{I_{max}[S]}{[S] + K_{m}}.
    \label{equation: Michaelis_menten}
\end{equation}

Since the concentrations of glucose and lactate being measured is low, such that $[S] << K_{m}$, equation~\ref{equation: Michaelis_menten} is reduced to a linear equation:

\begin{equation}
    I_{rate} = \frac{I_{max}[S]}{K_{m}}.
    \label{equation: reduced Michaelis_menten}
\end{equation}

\noindent Since the transimpedance amplifier converts the current to voltage, the relationship between voltage and concentration is a linear in the form of $y = mx + c$.

During calibration, only three known concentrations of solutions are used and the corresponding voltages recorded. Having few data points reduces the accuracy of the linear equation that best fits through the data points. Furthermore, there are numerous lines of best fit that would fit the data. Therefore, the optimal line of best fit can be found using least squares regression:

\begin{equation}
    m = \frac{ \sum_{i=1}^{3} \big( x_{i} - \Bar{X} \big) \big( y_{i} - \Bar{Y} \big) }{ \sum_{i=1}^{3} \big( x_{i} - \Bar{X} \big)^{2} },
    \label{equation: least squares regression}
\end{equation}
\begin{equation}
    c = \Bar{Y} - m\Bar{X},
\end{equation}

\noindent where $x_{i}$ is the voltage recording, $\Bar{X}$ is the average voltage recording, $y_i$ is the known concentration, and $\Bar{Y}$ is the average concentration from the data set. 

Now that the $m$ and $c$ coefficients have been determined, using $Concentration = m*Voltage + c$ the plot of voltage vs. time can be transposed into concentration vs. time for glucose and lactate.


% K+ derivation



